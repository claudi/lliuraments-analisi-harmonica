\documentclass[a4paper]{article}
\usepackage[T1]{fontenc}
\usepackage[utf8]{inputenc}
\usepackage[catalan]{babel}
\usepackage{amsmath}
\usepackage{amssymb}
\usepackage{lmodern}
\usepackage{microtype}

\newcommand{\iu}{\mathrm{i}}
\newcommand{\e}{\mathrm{e}}
\providecommand{\uppi}{\pi}
\newcommand{\diff}{\mathrm{d}}
\newcommand{\abs}[1]{\lvert{#1}\rvert}
\newcommand{\Abs}[1]{\left\lvert{#1}\right\rvert}
\newcommand{\conjugat}[1]{\overline{#1}}

\title{Primer lliurament}
\author{Claudi Lleyda Moltó -- 1462908}
\begin{document}
\maketitle

\begin{enumerate}
    \item[\textbf{1.}] Sigui~\(f\) una funció~\(2\uppi\)-periòdica, integrable
        Riemann sobre~\(\mathbb{R}\).
        \begin{enumerate}
            \item[\textbf{(a)}] Demostreu que la sèrie de Fourier de la
                funció~\(f\) es pot escriure com
                \[
                    f(\theta) \sim \widehat{f}(0) +
                    \sum_{n\geq1} \bigl[\widehat{f}(n) + \widehat{f}(-n)\bigr]
                    \cos(n\theta)
                    +
                    \iu\bigl[\widehat{f}(n)-\widehat{f}(-n)\bigr]\sin(n\theta).
                \]
        \end{enumerate}
\end{enumerate}
Sabem que podem escriure la sèrie de Fourier de~\(f\) com
\begin{align*}
    Sf(\theta) &= \sum_{n} \widehat{f}(n) \e^{\iu n\theta} \\
                 &= \widehat{f}(0)
                 + \sum_{n\geq1} \widehat{f}(n) \e^{\iu n\theta}
                 + \sum_{n\leq1} \widehat{f}(n) \e^{\iu n\theta}
                 \\
                 &= \widehat{f}(0)
                 + \sum_{n\geq1} \widehat{f}(n) \e^{\iu n\theta}
                 + \sum_{n\geq1} \widehat{f}(-n) \e^{-\iu n\theta} \\
                 &= \widehat{f}(0)
                 + \sum_{n\geq1} \widehat{f}(n)
                 \bigl[\cos(n\theta) + \iu\sin(n\theta)\bigr]
                 + \sum_{n\geq1} \widehat{f}(-n)
                 \bigl[\cos(n\theta) - \iu\sin(n\theta)\bigr] \\
                 &= \widehat{f}(0)
                 + \sum_{n\geq1}
                 \bigl[\widehat{f}(n)+\widehat{f}(-n)\bigr]\cos(n\theta)
                 + \iu\bigl[\widehat{f}(n)-\widehat{f}(-n)\bigr]\sin(n\theta)
\end{align*}
com volíem veure.

\begin{enumerate}
    \item[]\begin{enumerate}
        \item[\textbf{(b)}] Demostreu que si~\(f\) és parell,
            aleshores~\(\widehat{f}(n) = \widehat{f}(-n)\), i obtenim una sèrie
            de cosinus.
    \end{enumerate}
\end{enumerate}
Si~\(f\) és parell tenim, amb el canvi de variable~\(\theta = -\tau\), que
\begin{align*}
    \widehat{f}(n) &= \frac{1}{2\uppi}
                      \int_{-\uppi}^{\uppi}f(\theta)\e^{-\iu n\theta}\diff
                      \theta \\
                   &= \frac{-1}{2\uppi}
                      \int_{\uppi}^{-\uppi}f(-\tau)\e^{\iu n\tau}\diff
                      \tau \\
                   &= \frac{1}{2\uppi}
                      \int_{-\uppi}^{\uppi}f(-\tau)\e^{-\iu(-n)\tau}\diff
                      \tau \\
                   &= \widehat{f}(-n),
\end{align*}
i per tant~\(\widehat{f}(n) = \widehat{f}(-n)\), i de la fórmula que hem
demostrat a l'apartat anterior trobem que
\[
    Sf(\theta) = \widehat{f}(0) + 2\sum_{n\geq1}\widehat{f}(n)\cos(n\theta).
\]

\begin{enumerate}
    \item[]\begin{enumerate}
        \item[\textbf{(c)}] Demostreu que si~\(f\) és senar,
            aleshores~\(\widehat{f}(n) = -\widehat{f}(-n)\), i obtenim una sèrie
            de sinus.
    \end{enumerate}
\end{enumerate}
Si~\(f\) és senar tenim, amb el canvi de variable~\(\theta = -\tau\), que
\begin{align*}
    \widehat{f}(n) &= \frac{1}{2\uppi}
                      \int_{-\uppi}^{\uppi}f(\theta)\e^{-\iu n\theta}\diff
                      \theta \\
                   &= \frac{-1}{2\uppi}
                      \int_{\uppi}^{-\uppi}f(-\tau)\e^{\iu n\tau}\diff
                      \tau \\
                   &= \frac{-1}{2\uppi}
                      \int_{-\uppi}^{\uppi}f(\tau)\e^{-\iu(-n)\tau}\diff
                      \tau \\
                   &= -\widehat{f}(-n),
\end{align*}
i per tant~\(\widehat{f}(n) = -\widehat{f}(-n)\), i de la fórmula que hem
demostrat a l'apartat \textbf{(a)} trobem que
\[
    Sf(\theta) = 2\iu\sum_{n\geq1}\widehat{f}(n)\sin(n\theta).
\]

\begin{enumerate}
    \item[]\begin{enumerate}
        \item[\textbf{(d)}] Suposem que~\(f(\theta + \uppi) = f(\theta)\) per a
            tot~\(\theta\in\mathbb{R}\). Demostreu que~\(\widehat{f}(n) = 0\)
            per tot~\(n\) senar.
    \end{enumerate}
\end{enumerate}
Si~\(f\) satisfà~\(f(\theta + \uppi) = f(\theta)\) trobem que
\begin{align*}
    \widehat{f}(n) &= \frac{1}{2\uppi}
                      \int_{-\uppi}^{\uppi} f(\theta)\e^{-\iu n\theta}\diff
                      \theta \\
                   &= \frac{1}{2\uppi}
                      \int_{-\uppi}^{0} f(\theta)\e^{-\iu n\theta}\diff
                      \theta + \frac{1}{2\uppi}
                      \int_{0}^{\uppi} f(\theta)\e^{-\iu n\theta}\diff
                      \theta \\
                   &= \frac{1}{2\uppi}
                      \int_{0}^{\uppi} f(\theta-\uppi)
                      \e^{-\iu n(\theta-\uppi)}\diff
                      \theta + \frac{1}{2\uppi}
                      \int_{0}^{\uppi} f(\theta)\e^{-\iu n\theta}\diff
                      \theta \\
                   &= \frac{1}{2\uppi}
                      \int_{0}^{\uppi} f(\theta)
                      \e^{-\iu n\theta}
                      \e^{\iu n\uppi}
                      \diff\theta + \frac{1}{2\uppi}
                      \int_{0}^{\uppi} f(\theta)\e^{-\iu n\theta}\diff
                      \theta \\
                   &= \frac{1}{2\uppi}
                      \int_{0}^{\uppi}
                      f(\theta)
                      \e^{-\iu n\theta}
                      \e^{\iu n\uppi}
                      + f(\theta)\e^{-\iu n\theta}\diff
                      \theta \\
                   &= \frac{1}{2\uppi}
                      \int_{0}^{\uppi}
                      f(\theta)\bigl(1 + \e^{\iu n\uppi}\bigr)
                      \e^{-\iu n\theta}
                      \diff\theta \\
                   &= \frac{1}{2\uppi}
                      \int_{0}^{\uppi}
                      f(\theta)\bigl(1 + (-1)^{n}\bigr)
                      \e^{-\iu n\theta}
                      \diff\theta \\
                      \intertext{i quan~\(n\) és senar}
                   &= \frac{1}{2\uppi}
                      \int_{0}^{\uppi}
                      f(\theta)\bigl(1 - 1\bigr)
                      \e^{-\iu n\theta}
                      \diff\theta = 0,
\end{align*}
com volíem veure.

\begin{enumerate}
    \item[]\begin{enumerate}
        \item[\textbf{(e)}] Demostreu que~\(f\) és una funció real si i només
            si~\(\conjugat{\widehat{f}(n)} = \widehat{f}(-n)\) per a tota~\(n\).
    \end{enumerate}
\end{enumerate}

\clearpage
\begin{enumerate}
    \item[\textbf{2.}] A l'interval~\([-\uppi, \uppi]\) considereu la funció
        \[
            f(\theta) =
            \begin{cases}
                0 & \text{si } \abs{\theta} > \delta, \\
                1 - \abs{\theta}/\delta & \text{si } \abs{\theta} \leq \delta.
            \end{cases}
        \]
        Proveu que
        \[
            f(\theta) = \frac{\delta}{2\uppi}
            + 2 \sum_{n = 1}^{\infty}
            \frac{1-\cos(n\delta)}{n^{2}\uppi\delta} \cos(n\theta).
        \]
\end{enumerate}
Podem estendre~\(f\) a una funció~\(2\uppi\)-periòdica sobre
tot~\(\mathbb{R}\).
Denotem aquesta extensió per~\(f\), fent un abús de notació.

Calculem, per~\(n = 0\) trobem
\[
    \widehat{f}(0) = \frac{1}{2\uppi}
                     \int_{-\uppi}^{\uppi}f(\theta)\diff\theta
                   = \int_{-\delta}^{\delta}
                     \biggl(1 - \frac{\abs{\theta}}{\delta}\biggr)
                     \diff\delta
                   = \frac{\delta}{2\uppi}
\]
i per~\(n \neq 0\),
\begin{align*}
    \widehat{f}(n) &= \frac{1}{2\uppi} \int_{-\uppi}^{\uppi}
                      f(\theta) \e^{-\iu n\theta}
                      \diff \theta \\
                   &= \frac{1}{2\uppi} \int_{-\delta}^{\delta}
                      \bigl(1 - \abs{\theta}/\delta\bigr)
                      \e^{\iu n\theta}
                      \diff \theta \\
                   &= \frac{1}{2\uppi} \int_{-\delta}^{\delta}
                      \e^{\iu n\theta}
                      \diff \theta
                      - \frac{1}{2\uppi} \int_{-\delta}^{\delta}
                      \frac{\abs{\theta}}{\delta} \e^{\iu n\theta}
                      \diff \theta \\
                   &= \frac{\sin(n\delta)}{n\uppi}
                      - \frac{\cos(n\delta) - 1}{n^{2}\uppi\delta}
                      - \frac{\sin(n\delta)}{n\uppi} \\
                   &= \frac{1 - \cos(n\delta)}{n^{2}\uppi\delta}
\end{align*}

Observem que~\(f\) és parell. Per tant podem aplicar l'apartat~\(\textbf{(b)}\)
de l'exercici~\(\textbf{1.}\), i trobem doncs que la sèrie de Fourier de~\(f\)
és
\[
    Sf(\theta) = \frac{\delta}{2\uppi}
    + 2 \sum_{n = 1}^{\infty}
    \frac{1-\cos(n\delta)}{n^{2}\uppi\delta} \cos(n\theta).
\]

Per acabar, veiem que la sèrie de Fourier que hem trobat convergeix uniformement
a~\(f\). Observem que
\[
    \sum_{n=-\infty}^{\infty}\abs{\widehat{f}(n)}
    = \frac{\delta}{2\uppi} + \sum_{\substack{n=-\infty\\n\neq0}}^{\infty}
    \Abs{\frac{1 - \cos(n\delta)}{n^{2}\uppi\delta}}
    \sim \sum_{\substack{n=-\infty\\n\neq0}}^{\infty} \Abs{\frac{1}{n^{2}}}
    < \infty
\]
i per tant~\(Sf(\theta)\) convergeix uniformement a~\(f\), i tenim el que
volíem.

\clearpage
\begin{enumerate}
    \item[\textbf{3.}] Suposeu que~\(f\) és una funció periòdica de
        període~\(2\uppi\) que pertany a~\(\mathcal{C}^{k}\). Demostreu que
        \[
            \widehat{f}(n) = \mathrm{O}(1/\abs{n}^{k})
            \qquad\text{quan }
            \abs{n}\to\infty
        \]
\end{enumerate}
A classe hem vist la igualtat
\[
    \widehat{f'}(n) = 2\uppi\iu n\widehat{f}(n)
\]
i per tant podem veure que per a tota~\(m\leq k\) es satisfà
\[
    \widehat{f^{m)}}(n) = (2\uppi\iu n)^{m}\widehat{f}(n),
\]
d'on deduïm que
\begin{align*}
    \widehat{f}(n) &= \frac{\widehat{f^{k)}}(n)}{(2\uppi\iu n)^{k}} \\
                   &= \frac{\widehat{f^{k)}}(n)}{(2\uppi\iu)^{k}}
                      \frac{1}{n^{k}} \\
                   &\leq \frac{\widehat{f^{k)}}(n)}{(2\uppi)^{k}}
                      \frac{1}{\abs{n}^{k}}
\end{align*}
Pel lema de Riemann-Lebesgue tenim que~\(\widehat{f^{k)}}(n) \to 0\)
quan~\(n\to\infty\), i per tant per~\(n\) prou gran tindrem
que~\(\abs{\widehat{f^{k)}}(n)}\leq1\), d'on traiem que
\[
    \widehat{f}(n) \leq \frac{1}{(2\uppi)^{k}}\frac{1}{\abs{n}^{k}},
\]
i per tant~\(\widehat{f}(n) = \mathrm{O}(1/\abs{n}^{k})\)
quan~\(\abs{n}\to\infty\).

\end{document}

