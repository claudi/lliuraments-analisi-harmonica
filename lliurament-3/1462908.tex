\documentclass[a4paper]{article}
\usepackage[T1]{fontenc}
\usepackage[utf8]{inputenc}
\usepackage[catalan]{babel}
\usepackage{amsmath}
\usepackage{amssymb}
\usepackage{amsthm}
\usepackage{lmodern}
\usepackage{microtype}
\usepackage{url}


\theoremstyle{plain}
\newtheorem{theorem}{Teorema}

\newcommand{\iu}{\mathrm{i}}
\newcommand{\e}{\mathrm{e}}
\providecommand{\uppi}{\pi}
\newcommand{\diff}{\mathrm{d}}
\newcommand{\abs}[1]{\lvert{#1}\rvert}
\newcommand{\Abs}[1]{\left\lvert{#1}\right\rvert}
\newcommand{\F}{\mathcal{F}}
\newcommand{\D}{\mathcal{D}}
\newcommand{\Sc}{\mathcal{S}}
\newcommand{\conv}{\mathop{\ast}}

\title{Tercer lliurament}
\author{Claudi Lleyda Moltó -- 1462908}
\begin{document}
\maketitle

\begin{enumerate}
    \item[\textbf{1.}] Demostreu que si~\(f\) és contínua, de creixement moderat
        i
        \[
            \int_{-\infty}^{\infty}f(y)\e^{-y^{2}}\e^{2xy}\diff y=0
        \]
        per tot~\(x\in\mathbb{R}\), aleshores~\(f=0\).
\end{enumerate}

\begin{enumerate}
    \item[\textbf{2.}] Demostreu que la periodització del nucli de
        Fejér~\(\F_{N}\) a la recta real és igual al nucli de Fejér per funcions
        periòdiques de període~\(1\).
        En altres paraules,
        \[
            \sum_{n=-\infty}^{\infty} \F_{N}(x+n)
            = F_{N}(x),
        \]
        on~\(N\geq1\) és un enter, i on
        \[
            F_{N}(x) = \sum_{n=-N}^{N} \biggl(1 - \frac{\abs{n}}{N}\biggr)
            \e^{2\uppi\iu nx}
            = \frac{1}{N}\frac{\sin^{2}(N\uppi x)}{\sin^{2}(\uppi x)}.
        \]
\end{enumerate}

Per l'exercici~\textbf{3.} tenim que
\begin{align*}
    \sum_{n=-\infty}^{\infty} \F_{N}(x+n)
        &=
        \sum_{n=-\infty}^{\infty}
        N\biggl(\frac{\sin(N\uppi (x+n))}{N\uppi (x+n)}\biggr)^{2} \\
        &=
        \sum_{n=-\infty}^{\infty}
        N\biggl(\frac{\sin(N\uppi x)}{N\uppi (x+n)}\biggr)^{2} \\
        &=
        \frac{\sin^{2}(N\uppi x)}{N\uppi^{2}}
        \sum_{n=-\infty}^{\infty}
        \frac{1}{(n+x)^{2}} \\
        &=
        \frac{\sin^{2}(N\uppi x)}{N\uppi^{2}}
        \frac{\uppi^{2}}{\sin^{2}(\uppi x)}
        = \frac{1}{N}\frac{\sin^{2}(N\uppi x)}{\sin^{2}(\uppi x)}.
\end{align*}

\begin{theorem}[Fórmula de summació de Poisson]
    Sigui~\(f\in\Sc(\mathbb{R})\), aleshores
    \[
        \sum_{n=-\infty}^{\infty} f(x+n)
        = \sum_{n=-\infty}^{\infty} \widehat{f}(n) \e^{2\uppi\iu nx}.
    \]
\end{theorem}

\begin{enumerate}
    \item[\textbf{3.}] Aquest exemple dóna un altre exemple de periodització.
        \begin{enumerate}
            \item[\textbf{(a)}] Apliqueu la fórmula de sumació de Poisson a la
                funció~\(g\) del primer exercici del segon lliurament per
                obtenir
                \[
                    \sum_{n=-\infty}^{\infty} \frac{1}{(n+\alpha)^{2}}
                    = \frac{\uppi^{2}}{\sin^{2}(\uppi\alpha)}
                \]
                on~\(\alpha\) és real però no enter.
        \end{enumerate}
\end{enumerate}

Ja vam veure que si
\[
    g(x) = \begin{cases}
        1 - \abs{x} & \text{si } \abs{x} \leq 1 \\
        0 & \text{si no}
    \end{cases}
\]
aleshores
\[
    \widehat{g}(\xi) = \biggl(\frac{\sin(\uppi\xi)}{\uppi\xi}\biggr)^{2}.
\]

Per ser~\(g\in\Sc(\mathbb{R})\) tenim que la transformada de Fourier
de~\(\widehat{g}\) és~\(g\), i per tant en aplicar la fórmula de sumació de
Poisson trobem que
\[
    \sum_{n=-\infty}^{\infty}
    \frac{\sin^{2}(\uppi(\alpha+n))}{\uppi^{2}(\alpha+n)^{2}}
    =
    \sum_{n=-\infty}^{\infty}
    g(n) \e^{2\uppi\iu nx}
\]
Per un costat tenim que~\(g(n)=1\) si~\(n=0\), i~\(g(n)=0\) si~\(n\neq0\).
Per tant
\[
    \sum_{n=-\infty}^{\infty}
    g(n) \e^{2\uppi\iu nx}
    = 1.
\]
Per altra banda, tenim que~\(\sin(\uppi\alpha+\uppi n)=\sin(\uppi\alpha)\) per
tot~\(n\) enter, i així
\[
    \sum_{n=-\infty}^{\infty}
    \frac{\sin^{2}(\uppi(\alpha+n))}{\uppi^{2}(\alpha+n)^{2}}
    =
    \frac{\sin^{2}(\uppi\alpha)}{\uppi^{2}}
    \sum_{n=-\infty}^{\infty}
    \frac{1}{(\alpha+n)^{2}},
\]
i per tant
\[
    \sum_{n=-\infty}^{\infty} \frac{1}{(n+\alpha)^{2}}
    = \frac{\uppi^{2}}{\sin^{2}(\uppi\alpha)}
\]
sempre que~\(\sin(\uppi\alpha)\neq0\), i tenim prou amb que~\(\alpha\) no sigui
un enter.

\begin{enumerate}
    \item[]\begin{enumerate}
        \item[\textbf{(b)}] Demostreu com a conseqüència que
            \[
                \sum_{n=-\infty}^{\infty} \frac{1}{n+\alpha}
                = \frac{\uppi}{\tan(\uppi\alpha)}
            \]
            quan~\(\alpha\) és real però no enter.
    \end{enumerate}
\end{enumerate}

Si prenem~\(\alpha\) satisfent~\(0<\alpha<1\), aleshores les
funcions~\(f_{n}(\alpha)=\frac{1}{(n+\alpha)^{2}}\) són integrables, i per tant
\begin{multline*}
    \sum_{n=-\infty}^{\infty}
    \frac{-1}{n+\alpha}
    =
    \sum_{n=-\infty}^{\infty}
    \int
    \frac{1}{(n+\alpha)^{2}}
    \diff\alpha
    = \\ =
    \int
    \sum_{n=-\infty}^{\infty}
    \frac{1}{(n+\alpha)^{2}}
    \diff\alpha
    =
    \int
    \frac{\uppi^{2}}{\sin^{2}(\uppi\alpha)}
    \diff\alpha
    =
    \frac{-\uppi}{\tan(\uppi\alpha)}
\end{multline*}
i per tant
\[
    \sum_{n=-\infty}^{\infty} \frac{1}{n+\alpha}
    = \frac{\uppi}{\tan(\uppi\alpha)}.
\]

\begin{enumerate}
    \item[\textbf{4.}] El nucli de Dirichlet a la recta real està definit per
        \[
            \int_{-R}^{R}\widehat{f}(\xi)\e^{2\uppi\iu x\xi}\diff\xi
            = (f\ast \D_{R})(x)
        \]
        tal que
        \[
            \D_{R}(x) = \widehat{\chi_{[-R,R]}}(x)
            = \frac{\sin(2\uppi Rx)}{\uppi x}.
        \]

        A demés, el nucli de Dirichlet modificat per funcions de període~\(1\)
        està definit per
        \[
            \D^{\ast}_{N}(x)
            = \sum_{\abs{n} \leq N-1} \e^{2\uppi\iu nx}
            + \frac{1}{2}\bigl(\e^{-2\uppi\iu Nx} + \e^{2\uppi\iu Nx}\bigr).
        \]

        Demostreu que el resultat de l'exercici~\(\textbf{4.}\) ens dóna
        \[
            \sum_{n=-\infty}^{\infty} \D_{N}(x+n) = \D^{\ast}_{N}(x),
        \]
        on~\(N\geq1\) és un enter.
\end{enumerate}

\begin{enumerate}
    \item[\textbf{5.}] Apliqueu la fórmula de sumació de Poisson a
        \[
            f(x) = \frac{t}{\uppi(x^{2}+t^{2})}
            \qquad
            \text{i}
            \qquad
            \widehat{f}(\xi) = \e^{-2\uppi t\abs{\xi}}
        \]
        on~\(t>0\) per deduir
        \[
            \frac{1}{\uppi}
            \sum_{n=-\infty}^{\infty}
            \frac{t}{t^{2}+n^{2}}
            =
            \sum_{n=-\infty}^{\infty}
            \e^{-2\uppi t\abs{n}}.
        \]
\end{enumerate}

Per la fórmula de sumació de Poisson tenim que
\[
    \frac{1}{\uppi}
    \sum_{n=-\infty}^{\infty}
    \frac{t}{t^{2}+n^{2}}
    =
    \sum_{n=-\infty}^{\infty} f(n)
    =
    \sum_{n=-\infty}^{\infty} \widehat{f}(n)
    =
    \sum_{n=-\infty}^{\infty}
    \e^{-2\uppi t\abs{n}}.
\]
\end{document}

