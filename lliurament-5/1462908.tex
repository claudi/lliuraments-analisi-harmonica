\documentclass[a4paper]{article}
\usepackage[T1]{fontenc}
\usepackage[utf8]{inputenc}
\usepackage[catalan]{babel}
\usepackage{amsmath}
\usepackage{amssymb}
\usepackage{amsthm}
\usepackage{lmodern}
\usepackage{microtype}
\usepackage{url}

\theoremstyle{plain}
\newtheorem{theorem}{Teorema}

\theoremstyle{definition}
\newtheorem{definition}{Definició}

\DeclareMathOperator{\Lop}{L}

\newcommand{\iu}{\mathrm{i}}
\newcommand{\e}{\mathrm{e}}
\providecommand{\uppi}{\pi}
\newcommand{\diff}{\mathrm{d}}
\newcommand{\abs}[1]{\lvert{#1}\rvert}
\newcommand{\Abs}[1]{\left\lvert{#1}\right\rvert}
\newcommand{\F}{\mathcal{F}}
\newcommand{\D}{\mathcal{D}}
\newcommand{\Sc}{\mathcal{S}}
\newcommand{\conv}{\mathop{\ast}}

\newcommand{\ZZ}{\mathbb{Z}}
\newcommand{\RR}{\mathbb{R}}

\title{Cinquè lliurament}
\author{Claudi Lleyda Moltó -- 1462908}
\begin{document}
\maketitle

\begin{enumerate}
    \item[\textbf{1.}] Suposeu que~\(\psi\in\Sc(\RR^{d})\)
        satisfà~\(\int\abs{\psi(x)}^{2}=\diff x=1\). Demostreu que
        \[
            \biggl(\int_{\RR^{d}}\abs{x}^{2}\abs{\psi(x)}^{2}\diff x\biggr)
            \biggl(\int_{\RR^{d}}\abs{\xi}^{2}\abs{\widehat{\psi}(\xi)}^{2}\diff\xi\biggr)
            \geq
            \frac{d^{2}}{16\uppi^{2}}.
        \]
\end{enumerate}

\end{document}

