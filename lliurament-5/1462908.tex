\documentclass[a4paper]{article}
\usepackage[T1]{fontenc}
\usepackage[utf8]{inputenc}
\usepackage[catalan]{babel}
\usepackage{amsmath}
\usepackage{amssymb}
\usepackage{amsthm}
\usepackage{lmodern}
\usepackage{microtype}
\usepackage{url}

\theoremstyle{plain}
\newtheorem{theorem}{Teorema}

\theoremstyle{definition}
\newtheorem{definition}{Definició}

\let\Re\relax
\DeclareMathOperator{\Re}{Re}
\DeclareMathOperator{\X}{X}

\newcommand{\iu}{\mathrm{i}}
\newcommand{\e}{\mathrm{e}}
\providecommand{\uppi}{\pi}
\newcommand{\diff}{\mathrm{d}}
\newcommand{\abs}[1]{\lvert{#1}\rvert}
\newcommand{\Abs}[1]{\left\lvert{#1}\right\rvert}
\newcommand{\Sc}{\mathcal{S}}
\newcommand{\Hk}{\mathcal{H}}
\newcommand{\mlap}{-\Delta}
\newcommand{\conjugat}[1]{\overline{#1}}
\newcommand{\norm}[1]{\lVert{#1}\rVert}
\newcommand{\Norm}[1]{\left\lVert{#1}\right\rVert}

\newcommand{\RR}{\mathbb{R}}

\title{Cinquè lliurament}
\author{Claudi Lleyda Moltó -- 1462908}
\begin{document}
\maketitle

\begin{enumerate}
    \item[\textbf{1.}] Suposeu que~\(\psi\in\Sc(\RR^{d})\)
        satisfà~\(\int\abs{\psi(x)}^{2}=\diff x=1\). Demostreu que
        \[
            \biggl(\int_{\RR^{d}}\abs{x}^{2}\abs{\psi(x)}^{2}\diff x\biggr)
            \biggl(\int_{\RR^{d}}\abs{\xi}^{2}\abs{\widehat{\psi}(\xi)}^{2}\diff\xi\biggr)
            \geq
            \frac{d^{2}}{16\uppi^{2}}.
        \]
\end{enumerate}

Si considerem l'equació
\[
    \sum_{i=1}^{n}
    \frac{1}{2}
    x_{i}
    \frac{\partial}{\partial x_{i}}
    \abs{\psi(x)}^{2}
    =
    \Re\bigl(
        \nabla\psi\cdot\conjugat{x\psi(x)}
    \bigr)
\]
integrant sobre~\(\RR^{d}\) obtenim
\begin{align*}
    \frac{d}{2}
    \norm{\psi}_{2}^{2} &= \Re\biggl(
                           \int_{\RR^{d}}
                           \nabla\psi(x)\cdot\conjugat{x\psi(x)}
                           \diff x
                           \biggr) \\
                        &\leq \norm{\nabla\psi}_{2}
                           \norm{x\psi}_{2} \\
                        &= 2\uppi
                           \norm{\xi\widehat{\psi}}_{2}
                           \norm{x\psi}_{2},
\end{align*}
i tenim que
\[
    \norm{\xi\widehat{\psi}}_{2}
    \norm{x\psi}_{2}
    \geq
    \frac{d}{4\uppi}
\]
d'on deduïm
\[
    \biggl(\int_{\RR^{d}}\abs{x}^{2}\abs{\psi(x)}^{2}\diff x\biggr)
    \biggl(\int_{\RR^{d}}\abs{\xi}^{2}\abs{\widehat{\psi}(\xi)}^{2}\diff\xi\biggr)
    \geq
    \frac{d^{2}}{16\uppi^{2}}.
\]

\begin{enumerate}
    \item[\textbf{2.}] Considereu les equacions de calor a~\(\RR^{d}\) depenent
        del temps:
    \begin{equation}
        \label{eq:ex2:heat}
        \frac{\partial u}{\partial t}
        =
        \frac{\partial^{2}u}{\partial x_{1}^{2}}
        + \dots +
        \frac{\partial^{2}u}{\partial x_{d}^{2}},
        \qquad
        \text{on }
        t>0,
    \end{equation}
    amb condició de frontera~\(u(x,0)=f(x)\in\Sc(\RR^{d})\).
    Si
    \[
        \Hk_{t}^{(d)}
        =
        \frac{1}{(4\uppi t)^{d/2}}
        \e^{-\abs{x}^{2}/4t}
        =
        \int_{\RR^{d}}
        \e^{-4\uppi^{2}t\abs{\xi}^{2}}
        \e^{2\uppi\iu x\diff\xi}
        \diff\xi
    \]
    és el nucli de calor~\(d\)-dimensional, demostreu que la convolució
    \[
        u(x,t)
        =
        \bigl(f\ast\Hk_{t}^{(d)}\bigr)(x)
    \]
    és infinitament diferenciable quan~\(x\in\RR^{d}\) i~\(t>0\).

    A més,~\(u\) és solució de~\eqref{eq:ex2:heat},
    i és contínua fins a la frontera~\(t=0\),
    i satisfà la condició~\(u(x,0)=f(x)\).
\end{enumerate}

Tenim que
\[
    \frac{\partial^{k}}{\partial t^{k}}u(x,t)
    =
    \frac{\partial^{k}}{\partial t^{k}}
    \bigl(f\ast\Hk_{t}^{(d)}\bigr)(x)
    =
    \biggl(
        f\ast\frac{\partial^{k}}{\partial t^{k}}
        \Hk_{t}^{(d)}
    \biggr)(x),
\]
i donat que~\(\Hk_{t}^{(d)}\) és infinitament diferenciable quan~\(x\in\RR^{d}\)
i~\(t>0\), tenim que~\(u(x,t)\) també ho és.

\begin{enumerate}
    \item[\textbf{3.}] Per a cada~\((t,\theta)\) amb~\(t\in\RR\)
        i~\(\abs{\theta}<\uppi\), sigui~\(L=L_{t,\theta}\) la línia en el
        pla~\((x,y)\) determinada per
        \[
            x\cos(\theta) + y\sin(\theta) = t.
        \]
        Per~\(f\in\Sc(\RR^{d})\), la transformada de Radon dos dimensional
        de~\(f\) es defineix com
        \[
            \X(f)(t,\theta)
            =
            \int_{L_{t,\theta}}f
            =
            \int_{-\infty}^{\infty}
            f(t\cos(\theta) + u\sin(\theta), t\sin(\theta) - u\cos(\theta))
            \diff u
        \]
        Calculeu la transformada de Radon de la
        funció~\(f(x,y)=\e^{-\uppi(x^{2}+y^{2})}\).
\end{enumerate}

Calculem
\begin{align*}
    \X(f)(t,\theta) &= \int_{-\infty}^{\infty}
                       \e^{-\uppi
                           (t\cos(\theta) + u\sin(\theta))^{2}
                           +
                           (t\sin(\theta) - u\cos(\theta))^{2}}
                       \diff u \\
                    &= \int_{-\infty}^{\infty}
                       \e^{-\uppi
                           (t^{2}\cos^{2}(\theta) + u^{2}\sin^{2}(\theta)
                           +
                           t^{2}\sin^{2}(\theta) + u^{2}\cos^{2}(\theta))}
                       \diff u \\
                    &= \int_{-\infty}^{\infty}
                       \e^{-\uppi(t^{2} + u^{2})}
                       \diff u \\
                    &= \e^{-\uppi t^{2}}
\end{align*}

\begin{enumerate}
    \item[\textbf{4.}] Sigui~\(\X\) la transformada de Radon. Demostreu que
        si~\(f\in\Sc\) i~\(\X(f)=0\), aleshores~\(f=0\).
\end{enumerate}

\begin{enumerate}
    \item[\textbf{5.}] Per a tot real~\(a>0\), definim
        l'operador~\((\mlap)^{a}\) com
        \[
            (\mlap)^{a}f(x)
            =
            \int_{\RR^{d}}
            (2\uppi\abs{\xi})^{2a}
            \widehat{f}(\xi)
            \e^{2\uppi\iu\xi\cdot x}
            \diff\xi
        \]
        per a~\(f\in\Sc(\RR^{d})\).
        \begin{enumerate}
            \item[\textbf{(a)}] Demostreu que~\((\mlap)^{a}\) és equivalent a
                composar~\(\mlap\) (laplacià negatiu)~\(a\) vegades, on~\(a\)
                és un enter positiu.
        \end{enumerate}
\end{enumerate}

\begin{enumerate}
    \item[]\begin{enumerate}
        \item[\textbf{(b)}] Verifiqueu que~\((\mlap)^{a}(f)\) és infinitament
            diferenciable.
    \end{enumerate}
\end{enumerate}

\begin{enumerate}
    \item[]\begin{enumerate}
        \item[\textbf{(c)}] Demostreu que, en general, si~\(a\) no és un enter,
            aleshores~\((\mlap)^{a}(f)\) no és ràpidament decreixent.
    \end{enumerate}
\end{enumerate}

\end{document}

