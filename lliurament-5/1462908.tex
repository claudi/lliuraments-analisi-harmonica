\documentclass[a4paper]{article}
\usepackage[T1]{fontenc}
\usepackage[utf8]{inputenc}
\usepackage[catalan]{babel}
\usepackage{amsmath}
\usepackage{amssymb}
\usepackage{amsthm}
\usepackage{lmodern}
\usepackage{microtype}
\usepackage{url}

\theoremstyle{plain}
\newtheorem{theorem}{Teorema}

\theoremstyle{definition}
\newtheorem{definition}{Definició}

\newcommand{\iu}{\mathrm{i}}
\newcommand{\e}{\mathrm{e}}
\providecommand{\uppi}{\pi}
\newcommand{\diff}{\mathrm{d}}
\newcommand{\abs}[1]{\lvert{#1}\rvert}
\newcommand{\Abs}[1]{\left\lvert{#1}\right\rvert}
\newcommand{\Sc}{\mathcal{S}}
\newcommand{\Hk}{\mathcal{H}}

\newcommand{\RR}{\mathbb{R}}

\title{Cinquè lliurament}
\author{Claudi Lleyda Moltó -- 1462908}
\begin{document}
\maketitle

\begin{enumerate}
    \item[\textbf{1.}] Suposeu que~\(\psi\in\Sc(\RR^{d})\)
        satisfà~\(\int\abs{\psi(x)}^{2}=\diff x=1\). Demostreu que
        \[
            \biggl(\int_{\RR^{d}}\abs{x}^{2}\abs{\psi(x)}^{2}\diff x\biggr)
            \biggl(\int_{\RR^{d}}\abs{\xi}^{2}\abs{\widehat{\psi}(\xi)}^{2}\diff\xi\biggr)
            \geq
            \frac{d^{2}}{16\uppi^{2}}.
        \]
\end{enumerate}

\begin{enumerate}
    \item[\textbf{2.}] Considereu les equacions de calor a~\(\RR^{d}\) depenent
        del temps:
    \begin{equation}
        \label{eq:ex2:heat}
        \frac{\partial u}{\partial t}
        =
        \frac{\partial^{2}u}{\partial x_{1}^{2}}
        + \dots +
        \frac{\partial^{2}u}{\partial x_{d}^{2}},
        \qquad
        \text{on }
        t>0,
    \end{equation}
    amb condició de frontera~\(u(x,0)=f(x)\in\Sc(\RR^{d})\).
    Si
    \[
        \Hk_{t}^{(d)}
        =
        \frac{1}{(4\uppi t)^{d/2}}
        \e^{-\abs{x}^{2}/4t}
        =
        \int_{\RR^{d}}
        \e^{-4\uppi^{2}t\abs{\xi}^{2}}
        \e^{2\uppi\iu x\diff\xi}
        \diff\xi
    \]
    és el nucli de calor~\(d\)-dimensional, demostreu que la convolució
    \[
        u(x,t)
        =
        \bigl(f\ast\Hk_{t}^{(d)}\bigr)(x)
    \]
    és infinitament diferenciable quan~\(x\in\RR^{d}\) i~\(t>0\).

    A més,~\(u\) és solució de~\eqref{eq:ex2:heat},
    i és contínua fins a la frontera~\(t=0\),
    i satisfà la condició~\(u(x,0)=f(x)\).
\end{enumerate}

\end{document}

