\documentclass[a4paper]{article}
\usepackage[T1]{fontenc}
\usepackage[utf8]{inputenc}
\usepackage[catalan]{babel}
\usepackage{amsmath}
\usepackage{amssymb}
\usepackage{amsthm}
\usepackage{lmodern}
\usepackage{microtype}
\usepackage{url}


\theoremstyle{definition}
\newtheorem{definition}{Definició}

\newcommand{\iu}{\mathrm{i}}
\newcommand{\e}{\mathrm{e}}
\newcommand{\uppi}{\pi}
\newcommand{\diff}{\mathrm{d}}
\newcommand{\abs}[1]{\lvert{#1}\rvert}
\newcommand{\Abs}[1]{\left\lvert{#1}\right\rvert}
\newcommand{\F}{\mathcal{F}}
\newcommand{\conv}{\mathop{\ast}}

\title{Segon lliurament}
\author{Claudi Lleyda Moltó -- 1462908}
\begin{document}
\maketitle

\begin{enumerate}
    \item[\textbf{1.}] Siguin~\(f\) i~\(g\) dues funcions definides per
        \[
            f(x) = \chi_{[-1,1]}(x) = \begin{cases}
                1 & \text{si } \abs{x} \leq 1 \\
                0 & \text{si no}
            \end{cases}
            \qquad\text{i}\qquad
            g(x) = \begin{cases}
                1 - \abs{x} & \text{si } \abs{x} \leq 1 \\
                0 & \text{si no}
            \end{cases}
        \]
        Demostreu que
        \[
            \widehat{f}(\xi) = \frac{\sin(2\uppi\xi)}{\uppi\xi}
            \qquad\text{i}\qquad
            \widehat{g}(\xi) = \biggl(\frac{\sin(\uppi\xi)}{\uppi\xi}\biggr)^{2}
        \]
        sabent que~\(\widehat{f}(0) = 2\) i~\(\widehat{g}(0) = 1\).
\end{enumerate}

Si~\(\xi \neq 0\),
\begin{align*}
    \widehat{f}(\xi) &= \int_{-\infty}^{\infty}
                        f(x) \e^{2\uppi\iu x\xi}\diff x \\
                     &= \int_{-1}^{1}
                        \e^{2\uppi\iu x\xi} \diff x \\
                     &= \left[
                         \frac{\sin(2\uppi x\xi)}{2\uppi\xi}
                         - \frac{\iu\cos(2\uppi x\xi)}{2\uppi\xi}
                     \right]_{x = -1}^{1}\\
                     &= \frac{\sin(2\uppi\xi)}{\uppi\xi},
\end{align*}
i
\begin{align*}
    \widehat{g}(\xi) &= \int_{-\infty}^{\infty}
                        g(x) \e^{2\uppi\iu x\xi}\diff x \\
                     &= \int_{-1}^{1}
                        \bigl(1-\abs{x}\bigr) \e^{2\uppi\iu x\xi}\diff x \\
                     &= \int_{0}^{1}
                        \bigl(1-x\bigr) \e^{2\uppi\iu x\xi}\diff x
                      + \int_{-1}^{0}
                        \bigl(1+x\bigr) \e^{2\uppi\iu x\xi}\diff x \\
                     &= \frac{1-2\iu\uppi\xi-\e^{-2\iu\uppi\xi}}{4\uppi^{2}\xi^{2}}
                      + \frac{1+2\iu\uppi\xi-\e^{2\iu\uppi\xi}}{4\uppi^{2}\xi^{2}} \\
                     &= \frac{2-\e^{2\iu\uppi\xi}-\e^{-2\iu\uppi\xi}}{4\uppi^{2}\xi^{2}} \\
                     &= \biggl[
                        \frac{\e^{\iu\uppi\xi}-\e^{-\iu\uppi\xi}}{2\uppi\iu\xi}
                        \biggr]^{2}
                      = \biggl(\frac{\sin(\uppi\xi)}{\uppi\xi}\biggr)^{2}.
\end{align*}

\begin{definition}[Moderadament decreixent]
    Es diu que una funció~\(f\) sobre~\(\mathbb{R}\) és de decreixement moderat
    si existeix un~\(A>0\) tal que
    \[
        \abs{f(x)} \leq \Abs{\frac{A}{1+\abs{x}^{2}}}
        \qquad\text{per a tot }x\in\mathbb{R}.
    \]
\end{definition}

\begin{enumerate}
    \item[\textbf{2.}] El següent exercici i{\lgem}ustra el principi de que el
        decreixement de~\(\widehat{f}\) està relacionat amb les propietats de
        continuïtat de~\(f\).
        \begin{enumerate}
            \item[\textbf{(a)}] Suposem que~\(f\) és una funció de decreixement
                moderat sobre~\(\mathbb{R}\), la transformada de Fourier de la
                qual~\(\widehat{f}\) és contínua i satisfà
                \[
                    \widehat{f}(\xi) =
                    \mathrm{O}\biggl(\frac{1}{\abs{\xi}^{1+\alpha}}\biggr)
                    \qquad\text{quan } \abs{\xi}\to\infty
                \]
                per a cert~\(0<\alpha<1\). Demostreu que~\(f\) satisfà la
                condició de~H\"older d'ordre~\(\alpha\), és a dir, que
                \[
                    \abs{f(x+h)-f(x)} \leq M\abs{h}^{\alpha}
                    \qquad
                    \text{per a algun } M>0 \text{ i tot } x,h\in\mathbb{R}.
                \]
        \end{enumerate}
\end{enumerate}
Font: \url{https://math.stackexchange.com/a/2046706/457514}.

Si~\(h=0\) no cal fer cap argument.
Suposem doncs que~\(h\neq0\) i tenim que
\begin{align*}
    \abs{f(x+h) - f(x)} &= \Abs{\int_{-\infty}^{\infty}
                           \widehat{f}(\xi)\e^{2\uppi\iu(x+h)\xi}\diff\xi
                         - \int_{-\infty}^{\infty}
                           \widehat{f}(\xi)\e^{2\uppi\iu x\xi}\diff\xi} \\
                        &= \Abs{\int_{-\infty}^{\infty}
                           \widehat{f}(\xi)\e^{2\uppi\iu x\xi}
                           \bigl(\e^{2\uppi\iu h\xi}-1\bigr)
                           \diff\xi} \\
                        &\leq \int_{-\infty}^{\infty}
                           \abs{\widehat{f}(\xi)}\abs{e^{2\uppi\iu h\xi}-1}
                           \diff\xi \\
                        &= I_{1} + I_{2},
\end{align*}
on
\[
    I_{1} = \int_{\abs{\xi}\leq\abs{h}^{-1}}
    \abs{\widehat{f}(\xi)}\abs{e^{2\uppi\iu h\xi}-1} \diff\xi
    \qquad\text{i}\qquad
    I_{2} = \int_{\abs{\xi}\geq\abs{h}^{-1}}
    \abs{\widehat{f}(\xi)}\abs{e^{2\uppi\iu h\xi}-1} \diff\xi.
\]
i calculant trobem
\begin{gather*}
   \begin{split}
    I_{1} &= \int_{\abs{\xi}\leq\abs{h}^{-1}}
             \abs{\widehat{f}(\xi)}\abs{e^{2\uppi\iu h\xi}-1}
             \diff\xi \\
          &\leq \int_{\abs{\xi}\leq\abs{h}^{-1}}
             \frac{C}{\abs{\xi}^{1+\alpha}}
             \abs{e^{2\uppi\iu h\xi}-1}
             \diff\xi \\
          &\leq \int_{\abs{\xi}\leq\abs{h}^{-1}}
             \frac{C}{\abs{\xi}^{1+\alpha}}
             \abs{2\sin(\uppi h\xi)}
             \diff\xi \\
          &\leq 2\uppi C\abs{h}
             \int_{\abs{\xi}\leq\abs{h}^{-1}}
             \frac{1}{\abs{\xi}^{1+\alpha}}\abs{\xi}
             \diff\xi \\
          &= 4\uppi C\abs{h}
             \int_{0}^{\abs{h}^{-1}}
             \frac{\xi}{\xi^{1+\alpha}}
             \diff\xi \\
          &= 4\uppi C\frac{\abs{h}^{\alpha}}{1-\alpha}
    \end{split}
    \qquad\text{i}\qquad
    \begin{split}
    I_{2} &= \int_{\abs{\xi}\geq\abs{h}^{-1}}
             \abs{\widehat{f}(\xi)}\abs{e^{2\uppi\iu h\xi}-1}
             \diff\xi \\
          &\leq 2\int_{\abs{\xi}\geq\abs{h}^{-1}}
                \abs{\widehat{f}(\xi)}\diff\xi \\
          &\leq 2\int_{\abs{\xi}\geq\abs{h}^{-1}}
                \frac{C}{\abs{\xi}^{1+\alpha}} \diff\xi \\
          &= 2C
          \biggl[\frac{-\xi^{\alpha}}{\alpha}\biggr]_{\xi=\abs{h}^{-1}}^{\infty} \\
          &\leq 2C\frac{h^{\alpha}}{\alpha},
    \end{split}
\end{gather*}

\begin{enumerate}
    \item[]\begin{enumerate}
        \item[\textbf{(b)}] Sigui~\(f\) una funció contínua a~\(\mathbb{R}\) que
            s'anu{\lgem}a per~\(\abs{x}\geq1\), amb~\(f(0)=0\) i és igual
            a~\(1/\log(1/\abs{x})\) per a tot~\(x\) en un entorn de l'origen.
            Demostreu que~\(\widehat{f}\) no decreix moderadament.
            De fet, no existeix~\(\varepsilon>0\) tal que~\(\widehat{f}(\xi) =
            \textrm{O}(1/\abs{\xi}^{1+\varepsilon})\)
            quan~\(\abs{\xi}\to\infty\).
    \end{enumerate}
\end{enumerate}

Veiem que~\(f\) no és de H\"older.

\begin{enumerate}
    \item[\textbf{3.}] Suposeu que~\(f\) és contínua i de creixement moderat.
        \begin{enumerate}
            \item[\textbf{(a)}] Demostreu que~\(\widehat{f}\) és contínua
                i~\(\widehat{f}(\xi)\to0\) quan~\(\abs{\xi}\to\infty\).
        \end{enumerate}
\end{enumerate}

Comencem veient que~\(\widehat{f}\) és contínua.
Com per hipòtesi~\(f\) és de creixement moderat, tenim que existeix un~\(M>0\)
tal que~\(\abs{f(x)}<M\) per tot~\(x\in\mathbb{R}\), i si prenem
un~\(\varepsilon>0\) existeix un~\(K\in\mathbb{R}\) tal que
\[
    \int_{\abs{x}\geq K} \abs{f} \diff x < \varepsilon.
\]
Aleshores per a tot~\(t,\xi\in\mathbb{R}\), tenim
\[
    \abs{\widehat{f}(\xi + t)-\widehat{f}(\xi)}
    \leq
    \int_{-\infty}^{\infty}
    \abs{f(x)}
    \abs{e^{-2\uppi x\xi}}
    \abs{e^{-2\uppi\iu tx} - 1}
    \diff x
\]

\begin{enumerate}
    \item[]\begin{enumerate}
        \item[\textbf{(b)}] Demostreu que si~\(\widehat{f}(\xi) = 0\) per a
            tot~\(\xi\), aleshores~\(f\) és idènticament~\(0\).
    \end{enumerate}
\end{enumerate}

\begin{enumerate}
    \item[\textbf{4.}] La funció~\(\e^{-\uppi x^{2}}\) és la seva pròpia
        transformada de Fourier.
        Genereu altres funcions que (llevat d'una constant multiplicant) són les
        seves pròpies transformades de Fourier.
        Quines han de ser les constants multiplicant?
        Per dir això, demostreu que~\(\F^{4} = I\).
        Aquí~\(\F(f) = \widehat{f}\) és la transformada de Fourier,~\(\F^{4} =
        \F\circ\F\circ\F\circ\F\) i~\(I\) és l'operador identitat~\((If)(x) =
        f(x)\).
\end{enumerate}

\begin{enumerate}
    \item[\textbf{5.}] Si~\(f\) és de decreixement moderat, aleshores
        \begin{equation}
            \label{eq:ex9}
            \int_{-R}^{R}\biggl(1 - \frac{\abs{\xi}}{R}\biggr)
            \widehat{f}(\xi)\e^{2\uppi\iu x\xi}\diff\xi
            = (f\conv\F_{R})(x),
        \end{equation}
        on el nucli de Fejér a la recta real està definit per
        \[
            \F_{R}(t) = \begin{cases}\displaystyle
                R\biggl(\frac{\sin(\uppi tR)}{\uppi tR}\biggr)^{2}
                & \text{si } t \neq 0 \\
                R & \text{si } t = 0.
            \end{cases}
        \]
        Demostreu que~\(\{\F_{R}\}\) és una aproximació de la unitat
        quan~\(R\to\infty\), i per tant~\ref{eq:ex9} convergeix uniformement
        a~\(f(x)\) quan~\(R\to\infty\).
        Aquest és l'anàleg del teorema de Fejér per sèries de Fourier en el
        context de la transformada de Fourier.
\end{enumerate}

\end{document}

