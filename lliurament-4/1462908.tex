\documentclass[a4paper]{article}
\usepackage[T1]{fontenc}
\usepackage[utf8]{inputenc}
\usepackage[catalan]{babel}
\usepackage{amsmath}
\usepackage{amssymb}
\usepackage{amsthm}
\usepackage{lmodern}
\usepackage{microtype}
\usepackage{url}


\theoremstyle{plain}
\newtheorem{theorem}{Teorema}

\newcommand{\iu}{\mathrm{i}}
\newcommand{\e}{\mathrm{e}}
\providecommand{\uppi}{\pi}
\newcommand{\diff}{\mathrm{d}}
\newcommand{\abs}[1]{\lvert{#1}\rvert}
\newcommand{\Abs}[1]{\left\lvert{#1}\right\rvert}
\newcommand{\F}{\mathcal{F}}
\newcommand{\D}{\mathcal{D}}
\newcommand{\Sc}{\mathcal{S}}
\newcommand{\conv}{\mathop{\ast}}

\newcommand{\ZZ}{\mathbb{Z}}

\title{Quart lliurament}
\author{Claudi Lleyda Moltó -- 1462908}
\begin{document}
\maketitle

\begin{enumerate}
    \item[\textbf{1.}] Suposem que~\(f\) és de decreixement moderat i que la
        seva transformada de Fourier~\(\widehat{f}\) té suport~\(I=[-1/2,1/2]\).
        Aleshores,~\(f\) està completament determinada per la seva restricció
        a~\(\ZZ\).
        Això vol dir que si~\(g\) és un altre funció de decreixement moderat la
        transformada de Fourier de la qual té suport~\(I\) i~\(f(n)=g(n)\) per
        tot~\(n\in\ZZ\), aleshores~\(f=g\).
        Més precisament:
        \begin{enumerate}
            \item[\textbf{(a)}] Demostreu que la següent fórmula de reconstrucció
                és certa:
                \[
                    f(x) = \sum_{n=-\infty}^{\infty} f(n) K(x-n)
                    \qquad\text{on}\qquad
                    K(y) = \frac{\sin(\uppi y)}{\uppi y}.
                \]
        \end{enumerate}
\end{enumerate}

\begin{enumerate}
    \item[]\begin{enumerate}
        \item[\textbf{(b)}] Si~\(\lambda > 1\), aleshores
            \[
                f(x) =
                \sum_{n-\infty}^{\infty}
                \frac{1}{\lambda} f\Bigl(\frac{n}{\lambda}\Bigr)
                K_{\lambda}\Bigl(x-\frac{n}{\lambda}\Bigr)
                \quad\text{on}\quad
                K_{\lambda}(y) =
                \frac{\cos(\uppi y) - \cos(\uppi\lambda y)}
                {\uppi^{2}(\lambda-1)y^{2}}.
            \]
    \end{enumerate}
\end{enumerate}

\begin{enumerate}
    \item[]\begin{enumerate}
        \item[\textbf{(c)}] Demostreu que
            \[
                \int_{-\infty}^{\infty}
                \abs{f(x)}^{2}
                \diff x
                =
                \sum_{n=-\infty}^{\infty}
                \abs{f(n)}^{2}.
            \]
    \end{enumerate}
\end{enumerate}

\end{document}

