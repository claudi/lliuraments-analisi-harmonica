\documentclass[a4paper]{article}
\usepackage[T1]{fontenc}
\usepackage[utf8]{inputenc}
\usepackage[catalan]{babel}
\usepackage{amsmath}
\usepackage{amssymb}
\usepackage{amsthm}
\usepackage{lmodern}
\usepackage{microtype}
\usepackage{url}

\theoremstyle{plain}
\newtheorem{theorem}{Teorema}

\theoremstyle{definition}
\newtheorem{definition}{Definició}

\DeclareMathOperator{\Lop}{L}

\newcommand{\iu}{\mathrm{i}}
\newcommand{\e}{\mathrm{e}}
\providecommand{\uppi}{\pi}
\newcommand{\diff}{\mathrm{d}}
\newcommand{\abs}[1]{\lvert{#1}\rvert}
\newcommand{\Abs}[1]{\left\lvert{#1}\right\rvert}
\newcommand{\F}{\mathcal{F}}
\newcommand{\D}{\mathcal{D}}
\newcommand{\Sc}{\mathcal{S}}
\newcommand{\conv}{\mathop{\ast}}

\newcommand{\ZZ}{\mathbb{Z}}
\newcommand{\RR}{\mathbb{R}}

\title{Quart lliurament}
\author{Claudi Lleyda Moltó -- 1462908}
\begin{document}
\maketitle

\begin{enumerate}
    \item[\textbf{1.}] Suposem que~\(f\) és de decreixement moderat i que la
        seva transformada de Fourier~\(\widehat{f}\) té suport~\(I=[-1/2,1/2]\).
        Aleshores,~\(f\) està completament determinada per la seva restricció
        a~\(\ZZ\).
        Això vol dir que si~\(g\) és un altre funció de decreixement moderat la
        transformada de Fourier de la qual té suport~\(I\) i~\(f(n)=g(n)\) per
        tot~\(n\in\ZZ\), aleshores~\(f=g\).
        Més precisament:
        \begin{enumerate}
            \item[\textbf{(a)}] Demostreu que la següent fórmula de reconstrucció
                és certa:
                \[
                    f(x) = \sum_{n=-\infty}^{\infty} f(n) K(x-n)
                    \qquad\text{on}\qquad
                    K(y) = \frac{\sin(\uppi y)}{\uppi y}.
                \]
        \end{enumerate}
\end{enumerate}

Recordem que al primer exercici del segon lliurament vam veure
\[
    \widehat{\chi_{[-1,1]}}(\xi) = \frac{\sin(2\uppi \xi)}{\uppi \xi}.
\]

Aplicant la fórmula de sumació de Poisson a~\(\widehat{f}\) obtenim
\[
    \sum_{n=-\infty}^{\infty}
    \widehat{f}(\xi+n)
    =
    \sum_{n=-\infty}^{\infty}
    f(n)\e^{2\uppi\iu n\xi},
\]
i per tant
\begin{align*}
    \widehat{f}(\xi) &= \chi_{[-1/2,1/2]}(\xi)
                      \sum_{n=-\infty}^{\infty}
                      \widehat{f}(\xi+n) \\
                   &= \chi_{[-1/2,1/2]}(\xi)
                      \sum_{n=-\infty}^{\infty}
                      f(n)
                      \e^{2\uppi\iu n\xi} \\
                   &= \sum_{n=-\infty}^{\infty}
                      \chi_{[-1/2,1/2]}(\xi)
                      f(n)
                      \e^{2\uppi\iu n\xi},
\end{align*}
Aleshores, aplicant la transformada de Fourier trobem que
\begin{align*}
    f(x) &= \int_{-\infty}^{\infty}
            \sum_{n=-\infty}^{\infty}
            \chi_{[-1/2,1/2]}(t)
            f(n)
            \e^{2\uppi\iu nt}
            \e^{-2\uppi\iu x}
            \diff t \\
         &= \sum_{n=-\infty}^{\infty}
            f(n)
            \int_{-\infty}^{\infty}
            \chi_{[-1/2,1/2]}(t)
            \e^{2\uppi\iu nt}
            \e^{-2\uppi\iu x}
            \diff t \\
         &= \sum_{n=-\infty}^{\infty}
            f(n)
            \frac{\sin\bigl(\uppi(x-n)\bigr)}{\uppi(x-n)}
          = \sum_{n=-\infty}^{\infty}
            f(n)
            K(x-n).
\end{align*}

\begin{enumerate}
    \item[]\begin{enumerate}
        \item[\textbf{(b)}] Si~\(\lambda > 1\), aleshores
            \[
                f(x) =
                \sum_{n=-\infty}^{\infty}
                \frac{1}{\lambda} f\Bigl(\frac{n}{\lambda}\Bigr)
                K_{\lambda}\Bigl(x-\frac{n}{\lambda}\Bigr)
                \quad\text{on}\ %
                K_{\lambda}(y) =
                \frac{\cos(\uppi y) - \cos(\uppi\lambda y)}
                {\uppi^{2}(\lambda-1)y^{2}}.
            \]
    \end{enumerate}
\end{enumerate}

% https://www-users.math.umn.edu/~lerman/math5467/shannon_aliasing.pdf

\begin{enumerate}
    \item[]\begin{enumerate}
        \item[\textbf{(c)}] Demostreu que
            \[
                \int_{-\infty}^{\infty}
                \abs{f(x)}^{2}
                \diff x
                =
                \sum_{n=-\infty}^{\infty}
                \abs{f(n)}^{2}.
            \]
    \end{enumerate}
\end{enumerate}

Per la fórmula de Plancharel tenim que
\begin{align*}
    \int_{-\infty}^{\infty}
    \abs{f(x)}^{2}
    \diff x
    &=
    \int_{-\infty}^{\infty}
    \abs{\widehat{f}(\xi)}^{2}
    \diff \xi \\
    &=
    \int_{-1/2}^{1/2}
    \Abs{\sum_{n=-\infty}^{\infty}f(n)\e^{-2\uppi\iu nx}}^{2}
    \diff x \\
    &=
    \int_{-1/2}^{1/2}
    \sum_{m=-\infty}^{\infty}
    \sum_{n=-\infty}^{\infty}
    f(n)
    \overline{f(m)}
    \e^{-2\uppi\iu (n-m)x}
    \diff x \\
    &=
    \sum_{m=-\infty}^{\infty}
    \sum_{n=-\infty}^{\infty}
    f(n)
    \overline{f(m)}
    \int_{-1/2}^{1/2}
    \e^{-2\uppi\iu (n-m)x}
    \diff x \\
    &=
    \sum_{n=-\infty}^{\infty}
    \abs{f(n)}^{2}.
\end{align*}

\begin{enumerate}
    \item[\textbf{2.}] Suposem que~\(f\) és contínua en~\(\RR\).
        Demostreu que~\(f\) i~\(\widehat{f}\) no poden ser les dues de suport
        compacte a menys que~\(f=0\).
\end{enumerate}

Prenent~\(g(x)=f(\lambda x)\) tal que el suport de~\(g\) sigui~\([-1/2,1/2]\).
Aleshores per l'exercici anterior tenim que
\[
    g(x)
    =
    \sum_{n=-\infty}^{\infty}
    g(n)
    K(x-n)
    \qquad\text{on}\qquad
    K(x)
    =
    \frac{\sin(\uppi x)}{\uppi x}.
\]
Observem que quan~\(n\notin[-1/2,1/2]\) tindrem que~\(g(n)=0\), i per tant
aquesta és una suma finita, i deduïm~\(g\) és una funció entera, ja que la
funció~\(K(x)\) ho és.

Ara bé, pel principi de continuació analítica tenim que ha de ser~\(g(x)=0\),
i per tant ha de ser~\(f(x)=0\).

\begin{enumerate}
    \item[\textbf{3.}] Resoleu l'equació
        \[
            x^{2}\frac{\partial^{2}u}{\partial x^{2}}
            + ax\frac{\partial u}{\partial x}
            = \frac{\partial u}{\partial t}
        \]
        amb~\(u(x,0) = f(x)\) per~\(0<x<\infty\) i~\(t>0\) amb el canvi de
        variable~\(x=\e^{-y}\) tal que~\(-\infty<y<\infty\).
\end{enumerate}

Fent el canvi de variable de l'enunciat el problema es redueix a l'equació
\[
    \frac{\partial^{2} U}{\partial y^{2}}
    +
    (1 - a)
    \frac{\partial U}{\partial y}
    =
    \frac{\partial U}{\partial t}
\]
amb la condició
\begin{equation}
    \label{ex3:eq1}
    U(y,0) = F(y) = f(\e^{-y}).
\end{equation}

Aplicant la transformada de Fourier respecte~\(y\) trobem
\[
    (2\uppi\iu\xi)^{2}
    \widehat{U(\xi,t)}
    +
    (1 - a)
    (2\uppi\iu\xi)
    \widehat{U(\xi,t)}
    =
    \frac{\partial U}{\partial t}
    (\xi,t),
\]
que té per solució
\[
    \widehat{U}(\xi,t)
    =
    C(\xi)
    \e^{(-4\uppi^{2}\xi^{2}+(1-a)2\uppi\iu\xi)t},
\]
i imposant~\eqref{ex3:eq1} tenim
\[
    \widehat{U}(\xi,t)
    =
    F(\xi)
    \e^{-4\uppi^{2}\xi^{2}t+(1-a)2\uppi\iu\xi t}.
\]

Busquem solucions de la forma
\[
    U(y,t)
    =
    (F\ast K_{t})(y)
    \qquad\text{on}\qquad
    \widehat{K}_{t}(\xi)
    =
    \e^{-4\uppi^{2}\xi^{2}t+(1-a)2\uppi\iu\xi t}
\]
Calculem
\begin{align*}
    K_{t}(x) &= \int_{-\infty}^{\infty}
                \widehat{K}_{t}(\xi)
                \e^{2\uppi\iu\xi x}
                \diff\xi \\
             &= \int_{-\infty}^{\infty}
                \e^{-4\uppi^{2}\xi^{2}t+(1-a)2\uppi\iu\xi t}
                \e^{2\uppi\iu\xi x}
                \diff\xi \\
             &= \frac{1}{2\sqrt{\uppi t}}
             \e^{-{((1-a)t+x)^{2}}/{4t}}
\end{align*}

Per tant
\[
    U(y,t)
    =
    \frac{1}{2\sqrt{\uppi t}}
    \int_{-\infty}^{\infty}
    F(\tau)K_{t}
    (y-\tau)
    \diff\tau,
\]
i desfent el canvi de variable obtenim la solució
\[
    u(x,t)
    =
    \frac{1}{2\sqrt{\uppi t}}
    \int_{0}^{\infty}
    \e^{-(\log(v/x)+(1-a)t)^{2}/4t}
    \frac{f(v)}{v}
    \diff v.
\]

\begin{enumerate}
    \item[\textbf{4.}] Considereu l'equació~\(\Delta u=0\) a la banda
        \[
            \{(x,y):0<y<1, -\infty<x<\infty\}
        \]
        amb les condicions de frontera~\(u(x,0)=f_{0}(x)\)
        i~\(u(x,1)=f_{1}(x)\), on~\(f_{0}\) i~\(f(1)\) són ambdues de la classe
        de Schwartz.
        \begin{enumerate}
            \item[\textbf{(a)}] Demostreu que si~\(u\) és una solució a aquest
                problema, aleshores
                \[
                    \widehat{u}(\xi,y) =
                    A(\xi)\e^{2\uppi\xi y}
                    +
                    B(\xi)\e^{-2\uppi\xi y}.
                \]
                Expresseu~\(A\) i~\(B\) en termes de~\(\widehat{f}_{0}\)
                i~\(\widehat{f}_{1}\), i demostreu que
                \[
                    \widehat{u}(\xi,y) =
                    \frac{\sinh(2\uppi(1-y)\xi)}{\sinh(2\uppi\xi)}
                    \widehat{f}_{0}(\xi)
                    +
                    \frac{\sinh(2\uppi y\xi)}{\sinh(2\uppi\xi)}
                    \widehat{f}_{1}(\xi).
                \]
        \end{enumerate}
\end{enumerate}

\begin{enumerate}
    \item[]\begin{enumerate}
        \item[\textbf{(b)}] Demostreu com a resultat que
            \[
                \int_{-\infty}^{\infty}
                \abs{u(x,y) - f_{0}(x)}^{2}
                \diff x
                \to 0
                \qquad\text{quan}\qquad
                y\to 0
            \]
            i
            \[
                \int_{-\infty}^{\infty}
                \abs{u(x,y) - f_{1}(x)}^{2}
                \diff x
                \to 0
                \qquad\text{quan}\qquad
                y\to 1.
            \]
    \end{enumerate}
\end{enumerate}

\begin{enumerate}
    \item[]\begin{enumerate}
        \item[\textbf{(c)}] Si~\(\Phi(\xi)=\sinh(2\uppi a\xi)/\sinh(2\uppi\xi)\), amb
            amb~\(0\leq a<1\), aleshores~\(\Phi\) és la transformada de Fourier
            de~\(\varphi\) on
            \[
                \varphi(x)
                = \frac{\sin(\uppi a)}{2}
                \frac{1}{\cosh(\uppi x) + \cos(\uppi a)}.
            \]
    \end{enumerate}
\end{enumerate}

\begin{enumerate}
    \item[]\begin{enumerate}
        \item[\textbf{(d)}] Utilitzeu aquest resultat per expressar~\(u\) en
            termes d'integrals similars a les integrals de Poisson en funció
            de~\(f_{0}\) i~\(f_{1}\) de la següent manera:
            \begin{multline*}
                u(x,y)
                = \\
                \frac{\sin(\uppi y)}{2}
                \biggl(
                    \int_{-\infty}^{\infty}
                    \frac{f_{0}(x-t)}{\cosh(\uppi t)-\cos(\uppi y)}
                    \diff t
                    +
                    \int_{-\infty}^{\infty}
                    \frac{f_{1}(x-t)}{\cosh(\uppi t)+\cos(\uppi y)}
                    \diff t
                \biggr)
            \end{multline*}
    \end{enumerate}
\end{enumerate}

\begin{enumerate}
    \item[]\begin{enumerate}
        \item[\textbf{(e)}] Finalment, un pot veure que la funció~\(u(x,y)\)
            definida a l'apartat anterior és harmònica a la banda, i convergeix
            uniformement a~\(f_{0}(x\) quan~\(y\to0\) i a~\(f_{1}(x)\)
            quan~\(y\to1\).
            A més, un pot veure que~\(u(x,y)\) s'anu{\lgem}a a l'infinit. És a
            dir, que~\(\lim_{\abs{x}\to\infty}u(x,y)=0\) uniformement
            respecte~\(y\).
    \end{enumerate}
\end{enumerate}

\begin{enumerate}
    \item[\textbf{5.}] Les funcions d'Hermite~\(h_{k}(x)\) estan definides per
        la identitat generadora
        \[
            \sum_{k=0}^{\infty}
            h_{k}(x) \frac{t^{k}}{k!}
            =
            \e^{-(x^{2}/2 - 2tx+t^{2})}.
        \]
        \begin{enumerate}
            \item[\textbf{(a)}] Demostreu una definició alternativa de les
                funcions d'Hermite donada per la fórmula
                \[
                    h_{k}(x)
                    =
                    (-1)^{k} \e^{x^{2}/2}
                    \biggl(\frac{\diff}{\diff x}\biggr)^{k}
                    \e^{-x^{2}}.
                \]
                Concloeu que cada~\(h_{k}(x)\) és de la
                forma~\(P_{k}(x)\e^{-x^{2}/2}\), on~\(P_{k}\) és un polinomi de
                grau~\(k\).
                En particular, les funcions d'Hermite pertanyen a la classe de
                Schwartz i~\(h_{0}(x) =
                \e^{-x^{2}/2}\),~\(h_{1}(x)=2x\e^{-x^{2}/2}\).
        \end{enumerate}
\end{enumerate}

\begin{enumerate}
    \item[]\begin{enumerate}
        \item[\textbf{(b)}] Proveu que la família~\(\{h_{k}\}_{k=0}^{\infty}\)
            és completa en el sentit que si~\(f\) és de Schwartz i
            \[
                (f,h_{k}) = \int_{-\infty}^{\infty}f(x)h_{k}(x)\diff x = 0
                \qquad\text{per a tot}\qquad
                k\geq0
            \]
            aleshores~\(f=0\).
    \end{enumerate}
\end{enumerate}

\begin{enumerate}
    \item[]\begin{enumerate}
        \item[\textbf{(c)}]
            Definim~\(h^{\ast}_{k}(x)=h_{k}\bigl((2\uppi)^{1/2}x\bigr)\).
            Aleshores
            \[
                \widehat{h^{\ast}_{k}}(\xi) = (-\iu)^{k}h_{k}^{\ast}(\xi).
            \]
            Per tant, cada~\(h^{\ast}_{k}\) és una funció pròpia de la
            transformada de Fourier.
    \end{enumerate}
\end{enumerate}

\begin{definition}[Operador d'Hermite]
    Per a qualsevol funció~\(f\) de la classe de Schwartz definim l'operador
    \[
        \Lop(f) = -\frac{\diff^{2} f}{\diff x^{2}} + x^{2}f.
    \]
    Aquest operador s'anomena \emph{operador d'Hermite}.
\end{definition}

\begin{enumerate}
    \item[]\begin{enumerate}
        \item[\textbf{(d)}] Demostreu que~\(h_{k}\) és una funció pròpia de
            l'operador d'Hermite i proveu que
            \[
                \Lop(h_{k}) = (2k+1)h_{k}.
            \]
            En particular, concloem que les funcions~\(h_{k}\) són mútuament
            ortogonals pel producte escalar de~\(L^{2}\) a la classe de
            Schwartz.
    \end{enumerate}
\end{enumerate}

\begin{enumerate}
    \item[]\begin{enumerate}
        \item[\textbf{(e)}] Finalment, demostreu que
            \[
                \int_{-\infty}^{\infty}
                [h_{k}(x)]^{2}\diff x
                =
                \uppi^{1/2}2^{k}k!.
            \]
    \end{enumerate}
\end{enumerate}

\end{document}

