\documentclass[a4paper]{article}
\usepackage[T1]{fontenc}
\usepackage[utf8]{inputenc}
\usepackage[catalan]{babel}
\usepackage{amsmath}
\usepackage{amssymb}
\usepackage{amsthm}
\usepackage{lmodern}
\usepackage{microtype}
\usepackage{url}


\theoremstyle{plain}
\newtheorem{theorem}{Teorema}

\newcommand{\iu}{\mathrm{i}}
\newcommand{\e}{\mathrm{e}}
\providecommand{\uppi}{\pi}
\newcommand{\diff}{\mathrm{d}}
\newcommand{\abs}[1]{\lvert{#1}\rvert}
\newcommand{\Abs}[1]{\left\lvert{#1}\right\rvert}
\newcommand{\F}{\mathcal{F}}
\newcommand{\D}{\mathcal{D}}
\newcommand{\Sc}{\mathcal{S}}
\newcommand{\conv}{\mathop{\ast}}

\newcommand{\ZZ}{\mathbb{Z}}
\newcommand{\RR}{\mathbb{R}}

\title{Quart lliurament}
\author{Claudi Lleyda Moltó -- 1462908}
\begin{document}
\maketitle

\begin{enumerate}
    \item[\textbf{1.}] Suposem que~\(f\) és de decreixement moderat i que la
        seva transformada de Fourier~\(\widehat{f}\) té suport~\(I=[-1/2,1/2]\).
        Aleshores,~\(f\) està completament determinada per la seva restricció
        a~\(\ZZ\).
        Això vol dir que si~\(g\) és un altre funció de decreixement moderat la
        transformada de Fourier de la qual té suport~\(I\) i~\(f(n)=g(n)\) per
        tot~\(n\in\ZZ\), aleshores~\(f=g\).
        Més precisament:
        \begin{enumerate}
            \item[\textbf{(a)}] Demostreu que la següent fórmula de reconstrucció
                és certa:
                \[
                    f(x) = \sum_{n=-\infty}^{\infty} f(n) K(x-n)
                    \qquad\text{on}\qquad
                    K(y) = \frac{\sin(\uppi y)}{\uppi y}.
                \]
        \end{enumerate}
\end{enumerate}

\begin{enumerate}
    \item[]\begin{enumerate}
        \item[\textbf{(b)}] Si~\(\lambda > 1\), aleshores
            \[
                f(x) =
                \sum_{n-\infty}^{\infty}
                \frac{1}{\lambda} f\Bigl(\frac{n}{\lambda}\Bigr)
                K_{\lambda}\Bigl(x-\frac{n}{\lambda}\Bigr)
                \quad\text{on}\quad
                K_{\lambda}(y) =
                \frac{\cos(\uppi y) - \cos(\uppi\lambda y)}
                {\uppi^{2}(\lambda-1)y^{2}}.
            \]
    \end{enumerate}
\end{enumerate}

\begin{enumerate}
    \item[]\begin{enumerate}
        \item[\textbf{(c)}] Demostreu que
            \[
                \int_{-\infty}^{\infty}
                \abs{f(x)}^{2}
                \diff x
                =
                \sum_{n=-\infty}^{\infty}
                \abs{f(n)}^{2}.
            \]
    \end{enumerate}
\end{enumerate}

\begin{enumerate}
    \item[\textbf{2.}] Suposem que~\(f\) és contínua en~\(\RR\).
        Demostreu que~\(f\) i~\(\widehat{f}\) no poden ser les dues de suport
        compacte a menys que~\(f=0\).
\end{enumerate}

\begin{enumerate}
    \item[\textbf{3.}] Resoleu l'equació
        \[
            x^{2}\frac{\partial^{2}u}{\partial x^{2}}
            + ax\frac{\partial u}{\partial x}
            = \frac{\partial u}{\partial t}
        \]
        amb~\(u(x,0) = f(x)\) per~\(0<x<\infty\) i~\(t>0\) amb el canvi de
        variable~\(x=\e^{-y}\) tal que~\(-\infty<y<\infty\).
\end{enumerate}

\begin{enumerate}
    \item[\textbf{4.}] Considereu l'equació~\(\Delta u=0\) a la banda
        \[
            \{(x,y):0<y<1, -\infty<x<\infty\}
        \]
        amb les condicions de frontera~\(u(x,0)=f_{0}(x)\)
        i~\(u(x,1)=f_{1}(x)\), on~\(f_{0}\) i~\(f(1)\) són ambdues de la classe
        de Schwartz.
        \begin{enumerate}
            \item[\textbf{(a)}] Demostreu que si~\(u\) és una solució a aquest
                problema, aleshores
                \[
                    \widehat{u}(\xi,y) =
                    A(\xi)\e^{2\uppi\xi y}
                    +
                    B(\xi)\e^{-2\uppi\xi y}.
                \]
                Expresseu~\(A\) i~\(B\) en termes de~\(\widehat{f}_{0}\)
                i~\(\widehat{f}_{1}\), i demostreu que
                \[
                    \widehat{u}(\xi,y) =
                    \frac{\sinh(2\uppi(1-y)\xi)}{\sinh(2\uppi\xi)}
                    \widehat{f}_{0}(\xi)
                    +
                    \frac{\sinh(2\uppi y\xi)}{\sinh(2\uppi\xi)}
                    \widehat{f}_{0}(\xi).
                \]
        \end{enumerate}
\end{enumerate}

\begin{enumerate}
    \item[]\begin{enumerate}
        \item[\textbf{(b)}] Demostreu com a resultat que
            \[
                \int_{-\infty}^{\infty}
                \abs{u(x,y) - f_{0}(x)}^{2}
                \diff x
                \to 0
                \qquad\text{quan}\qquad
                y\to 0
            \]
            i
            \[
                \int_{-\infty}^{\infty}
                \abs{u(x,y) - f_{1}(x)}^{2}
                \diff x
                \to 0
                \qquad\text{quan}\qquad
                y\to 1.
            \]
    \end{enumerate}
\end{enumerate}

\begin{enumerate}
    \item[]\begin{enumerate}
        \item[\textbf{(c)}] Si~\(\Phi(\xi)=\sinh(2\uppi a\xi)/\sinh(2\uppi\xi)\), amb
            amb~\(0\leq a<1\), aleshores~\(\Phi\) és la transformada de Fourier
            de~\(\varphi\) on
            \[
                \varphi(x)
                = \frac{\sin(\uppi a)}{2}
                \frac{1}{\cosh(\uppi x) + \cos(\uppi a)}.
            \]
    \end{enumerate}
\end{enumerate}

\begin{enumerate}
    \item[]\begin{enumerate}
        \item[\textbf{(d)}] Utilitzeu aquest resultat per expressar~\(u\) en
            termes d'integrals similars a les integrals de Poisson en funció
            de~\(f_{0}\) i~\(f_{1}\) de la següent manera:
            \begin{multline*}
                u(x,y)
                = \\
                \frac{\sin(\uppi y)}{2}
                \biggl(
                    \int_{-\infty}^{\infty}
                    \frac{f_{0}(x-t)}{\cosh(\uppi t)-\cos(\uppi y)}
                    \diff t
                    +
                    \int_{-\infty}^{\infty}
                    \frac{f_{1}(x-t)}{\cosh(\uppi t)+\cos(\uppi y)}
                    \diff t
                \biggr)
            \end{multline*}
    \end{enumerate}
\end{enumerate}

\begin{enumerate}
    \item[]\begin{enumerate}
        \item[\textbf{(e)}] Finalment, un pot veure que la funció~\(u(x,y)\)
            definida a l'apartat anterior és harmònica a la banda, i convergeix
            uniformement a~\(f_{0}(x\) quan~\(y\to0\) i a~\(f_{1}(x)\)
            quan~\(y\to1\).
            A més, un pot veure que~\(u(x,y)\) s'anu{\lgem}a a l'infinit. És a
            dir, que~\(\lim_{\abs{x}\to\infty}u(x,y)=0\) uniformement
            respecte~\(y\).
    \end{enumerate}
\end{enumerate}

\begin{enumerate}
    \item[\textbf{5.}] Les funcions d'Hermite~\(h_{k}(x)\) estan definides per
        la identitat generadora
        \[
            \sum_{k=0}^{\infty}
            h_{k}(x) \frac{t^{k}}{k!}
            =
            \e^{-(x^{2}/2 - 2tx+t^{2})}.
        \]
        \begin{enumerate}
            \item[\textbf{(a)}] Demostreu una definició alternativa de les
                funcions d'Hermite donada per la fórmula
                \[
                    h_{k}(x)
                    =
                    (-1)^{k} \e^{x^{2}/2}
                    \biggl(\frac{\diff}{\diff x}\biggr)^{k}
                    \e^{-x^{2}}.
                \]
                Concloeu que cada~\(h_{k}(x)\) és de la
                forma~\(P_{k}(x)\e^{-x^{2}/2}\), on~\(P_{k}\) és un polinomi de
                grau~\(k\).
                En particular, les funcions d'Hermite pertanyen a la classe de
                Schwartz i~\(h_{0}(x) =
                \e^{-x^{2}/2}\),~\(h_{1}(x)=2x\e^{-x^{2}/2}\).
        \end{enumerate}
\end{enumerate}

\begin{enumerate}
    \item[]\begin{enumerate}
        \item[\textbf{(c)}]
            Definim~\(h^{\ast}_{k}(x)=h_{k}\bigl((2\uppi)^{1/2}x\bigr)\).
            Aleshores
            \[
                \widehat{h^{\ast}_{k}}(\xi) = (-\iu)^{k}h_{k}^{\ast}(\xi).
            \]
            Per tant, cada~\(h^{\ast}_{k}\) és una funció pròpia de la
            transformada de Fourier
    \end{enumerate}
\end{enumerate}

\end{document}

