\documentclass[a4paper]{article}
\usepackage[T1]{fontenc}
\usepackage[utf8]{inputenc}
\usepackage[catalan]{babel}
\usepackage{amsmath}
\usepackage{amssymb}
\usepackage{amsthm}
\usepackage{lmodern}
\usepackage{microtype}
\usepackage{url}

\theoremstyle{plain}
\newtheorem{theorem}{Teorema}

\theoremstyle{definition}
\newtheorem{definition}{Definició}

\newcommand{\iu}{\mathrm{i}}
\newcommand{\e}{\mathrm{e}}
\providecommand{\uppi}{\pi}
\newcommand{\diff}{\mathrm{d}}
\newcommand{\abs}[1]{\lvert{#1}\rvert}
\newcommand{\Abs}[1]{\left\lvert{#1}\right\rvert}

\newcommand{\RR}{\mathbb{R}}
\newcommand{\ZZ}{\mathbb{Z}}

\title{Sisè lliurament}
\author{Claudi Lleyda Moltó -- 1462908}
\begin{document}
\maketitle

\begin{enumerate}
    \item[\textbf{1.}] Si~\(f\) és una funció~\(\mathcal{C}^{1}\) sobre el
        cercle, demostreu que~\(\abs{a_{N}(n)}\leq c/\abs{n}\)
        quan~\(0<\abs{n}\leq N/2\).
\end{enumerate}

\begin{enumerate}
    \item[\textbf{2.}] Suposeu
        que~\(\displaystyle P(x)=\sum_{n=1}^{N}a_{n}\e^{2\uppi\iu nx}\).
        \begin{enumerate}
            \item[\textbf{(a)}] Demostreu utilitzant les identitats de Parseval
                al cercle i~\(\ZZ(N)\), que
                \[
                    \int_{0}^{1}
                    \abs{P(x)}^{2}
                    \diff x
                    =
                    \frac{1}{N}
                    \sum_{j=1}^{N}
                    \abs{P(j/N)}^{2}.
                \]
        \end{enumerate}
\end{enumerate}

\begin{enumerate}
    \item[]\begin{enumerate}
            \item[\textbf{(b)}] Demostreu la fórmula de reconstrucció
                \[
                    P(x)
                    =
                    \sum_{j=1}^{N}
                    P(j/N)
                    K\bigl(x-(j/N)\bigr)
                \]
                on
                \[
                    K(x)
                    =
                    \frac{\e^{2\uppi\iu x}}{N}
                    \frac{1-\e^{2\uppi\iu Nx}}{1-\e^{2\uppi\iu x}}
                    =
                    \frac{1}{N}
                    \bigl(
                        e^{2\uppi\iu x}
                        +
                        \e^{2\uppi\iu2x}
                        + \cdots +
                        \e^{2\uppi\iu Nx}
                    \bigr).
                \]
        \end{enumerate}
\end{enumerate}

\begin{enumerate}
    \item[\textbf{3.}] Demostreu les següents afirmacions:
        \begin{enumerate}
            \item[\textbf{(a)}] Demostreu que es pot calcular els coeficients de
                Fourier d'una funció a~\(\ZZ(N)\) quan~\(N=3^{n}\) amb, com a
                molt,~\(6N\log_{3}(N)\) operacions.
        \end{enumerate}
\end{enumerate}

\begin{enumerate}
    \item[]\begin{enumerate}
        \item[\textbf{(b)}] Generalitzeu aquest fet per~\(N=\alpha^{n}\)
            on~\(\alpha\) és un enter més gran que~\(1\).
    \end{enumerate}
\end{enumerate}

\begin{enumerate}
    \item[\textbf{4.}] Escriviu les taules de multiplicació dels
        grups~\(\ZZ^{\ast}(3)\),~\(\ZZ^{\ast}(4)\),~\(\ZZ^{\ast}(5)\),~\(\ZZ^{\ast}(6)\),
        \(\ZZ^{\ast}(8)\) i~\(\ZZ^{\ast}(9)\).
        Quins d'aquests grups són cíclics?
\end{enumerate}
\end{document}

