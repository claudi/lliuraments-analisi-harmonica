\documentclass[a4paper]{article}
\usepackage[T1]{fontenc}
\usepackage[utf8]{inputenc}
\usepackage[catalan]{babel}
\usepackage{amsmath}
\usepackage{amssymb}
\usepackage{amsthm}
\usepackage{lmodern}
\usepackage{microtype}
\usepackage{url}

\theoremstyle{plain}
\newtheorem{theorem}{Teorema}

\theoremstyle{definition}
\newtheorem{definition}{Definició}

\newcommand{\iu}{\mathrm{i}}
\newcommand{\e}{\mathrm{e}}
\providecommand{\uppi}{\pi}
\newcommand{\diff}{\mathrm{d}}
\newcommand{\abs}[1]{\lvert{#1}\rvert}
\newcommand{\Abs}[1]{\left\lvert{#1}\right\rvert}
\newcommand{\conjugat}[1]{\overline{#1}}

\newcommand{\RR}{\mathbb{R}}
\newcommand{\ZZ}{\mathbb{Z}}

\title{Sisè lliurament}
\author{Claudi Lleyda Moltó -- 1462908}
\begin{document}
\maketitle

\begin{enumerate}
    \item[\textbf{1.}] Si~\(f\) és una funció~\(\mathcal{C}^{1}\) sobre el
        cercle, demostreu que~\(\abs{a_{N}(n)}\leq c/\abs{n}\)
        quan~\(0<\abs{n}\leq N/2\).
\end{enumerate}

Sigui~\(l\in\ZZ\setminus\{0\}\) tenim que
\begin{align*}
    a_{N}(n)
    \bigl(
        1 - \e^{2\uppi\iu ln/N}
    \bigr)
    &=
    \bigl(
        1 - \e^{2\uppi\iu ln/N}
    \bigr)
    \frac{1}{N}
    \sum_{k=1}^{N}
    f\bigl(
        \e^{2\uppi\iu k/N}
    \bigr)
    \e^{-2\uppi\iu kn/N} \\
    &=
    \frac{1}{N}
    \sum_{k=1}^{N}
    f\bigl(
        \e^{2\uppi\iu k/N}
    \bigr)
    \e^{-2\uppi\iu kn/N}
    -
    \frac{1}{N}
    \sum_{k=1}^{N}
    f\bigl(
        \e^{2\uppi\iu k/N}
    \bigr)
    \e^{2\uppi\iu\frac{l-k}{N}n},
\end{align*}
i prenent~\(-t=l-k\), trobem
\[
    a_{N}(n)
    =
    \frac{1}{N}
    \sum_{k=1}^{N}
    f\bigl(
        \e^{2\uppi\iu k/N}
    \bigr)
    \e^{-2\uppi\iu kn/N}
    -
    \frac{1}{N}
    \sum_{t=1-l}^{N-l}
    f\bigl(
        \e^{2\uppi\iu(l+t)/N}
    \bigr)
    \e^{2\uppi\iu tn/N}.
\]

Ara bé, la funció exponencial és~\(1\)-periòdica, així que també serà~\(l\)
periòdica, i per tant
\[
    a_{N}(n)
    \bigl(
        1 - \e^{2\uppi\iu ln/N}
    \bigr)
    =
    \frac{1}{N}
    \sum_{k=1}^{N}
    \Bigl(
        f\bigl(
            \e^{2\uppi\iu k/N}
        \bigr)
        -
        f\bigl(
            \e^{2\uppi\iu(k+l)/N}
        \bigr)
    \Bigr)
    \e^{-2\uppi\iu kn/N}.
\]

Si prenem valors absoluts tenim, per la condició de Lipschitz,
\begin{align*}
    \abs{a_{N}(n)}
    \Abs{1-\e^{2\uppi\iu ln/N}}
    &=
    \frac{1}{N}
    \Abs{
        \sum_{k=1}^{N}
        \Bigl(
            f\bigl(
                \e^{2\uppi\iu k/N}
            \bigr)
            -
            f\bigl(
                \e^{2\uppi\iu(k+l)/N}
            \bigr)
        \Bigr)
        \e^{-2\uppi\iu kn/N}
    } \\
    &\leq
    \frac{1}{N}
    \Abs{
        \sum_{k=1}^{N}
        \Bigl(
            f\bigl(
                \e^{2\uppi\iu k/N}
            \bigr)
            -
            f\bigl(
                \e^{2\uppi\iu(k+l)/N}
            \bigr)
        \Bigr)
    }
    \Abs{
        \e^{-2\uppi\iu kn/N}
    } \\
    &\leq
    \frac{M}{N}
    \Abs{
        \sum_{k=1}^{N}
        \Bigl(
            \e^{2\uppi\iu k/N}
            -
            \e^{2\uppi\iu(k+l)/N}
        \Bigr)
    } \\
    &=
    \frac{M}{N}
    \sum_{k=1}^{N}
    \Abs{\e^{2\uppi\iu k/N}}
    \Abs{1-\e^{2\uppi\iu l/N}} \\
    &=
    \frac{MN}{N}
    \Abs{\e^{2\uppi\iu k/N}},
\end{align*}
per a certa~\(M\),
i per tant
\[
    \abs{a_{N}(n)}
    \leq
    M
    \frac{\abs{1-\e^{2\uppi\iu l/N}}}{\abs{1-\e^{2\uppi\iu ln/N}}}.
\]

Si triem~\(l\) tal que
\[
    \Abs{l-\frac{N}{2n}}
    \leq
    \frac{1}{2}
\]
tindrem
\[
    \Abs{\frac{ln}{N} - \frac{1}{2}}
    \leq
    \frac{\abs{n}}{2N},
\]
d'on deduïm
\[
    \Abs{\frac{ln}{N} - \frac{1}{2}}
    \leq
    \frac{N/2}{2N}
    =
    \frac{1}{4},
\]
i per tant
\[
    \frac{1}{4}
    \leq
    \frac{ln}{N}
    \leq
    \frac{3}{4}.
\]

Acotem~\(a_{N}(n)\). Tenim que
\begin{align*}
    \Abs{1-\e^{2\uppi\iu l/N}}
    &\leq \abs{2\uppi\iu l/N} \\
    &\leq 2\uppi\Abs{\frac{1}{N}}\Abs{\frac{1}{2} + \frac{N}{2\uppi\iu}} \\
    &\leq 2\uppi\biggl(
        \frac{1}{2N} + \frac{1}{2\abs{n}}
    \biggr) \\
    &\leq 2\uppi\biggl(
        \frac{2}{2\abs{n}} + \frac{1}{2\abs{n}}
    \biggr) \\
    &= \frac{3\uppi}{\abs{n}}
\end{align*}
i
\begin{align*}
    \Abs{1-\e^{2\uppi\iu ln/N}}
    &\geq \Abs{1-\e^{2\uppi\iu 3/4}} \\
    &= \Abs{1-\e^{3\uppi\iu/2}} \\
    &= \abs{1+\iu} = \sqrt{2},
\end{align*}
i per tant
\[
    \abs{a_{N}(n)} \leq \frac{3M\uppi}{\sqrt{2}\abs{n}} = \frac{c}{\abs{n}},
\]
on~\(c=3M\uppi/\sqrt{2}\).

\clearpage

\begin{enumerate}
    \item[\textbf{2.}] Suposeu
        que~\(\displaystyle P(x)=\sum_{n=1}^{N}a_{n}\e^{2\uppi\iu nx}\).
        \begin{enumerate}
            \item[\textbf{(a)}] Demostreu utilitzant les identitats de Parseval
                al cercle i~\(\ZZ(N)\), que
                \[
                    \int_{0}^{1}
                    \abs{P(x)}^{2}
                    \diff x
                    =
                    \frac{1}{N}
                    \sum_{j=1}^{N}
                    \abs{P(j/N)}^{2}.
                \]
        \end{enumerate}
\end{enumerate}

Tenim que
\begin{align*}
    \int_{0}^{1}
    \abs{P(x)}^{2}
    \diff x
    &=
    \int_{0}^{1}
    \Abs{
        \sum_{n=1}^{N}
        a_{n}\e^{2\uppi\iu nx}
    }^{2}
    \diff x \\
    &=
    \sum_{n=1}^{N}
    \int_{0}^{1}
    \abs{a_{n}}^{2}
    \Abs{
        \e^{2\uppi\iu nx}
    }^{2}
    \diff x \\
    &=
    \sum_{n=1}^{N}
    \int_{0}^{1}
    \abs{a_{n}}^{2}
    \diff x \\
    &=
    \sum_{n=1}^{N}
    \abs{a_{n}}^{2}.
\end{align*}

Considerem ara l'aplicació
\[
    Q(\xi_{j}) = P(j/N),
    \qquad
    \text{on}
    \quad
    \xi_{j} = \e^{2\uppi j/N}.
\]
Aleshores
\[
    Q(\xi_{j})
    =
    P(j/N)
    =
    \sum_{n=1}^{N}
    a_{n}
    \e^{2\uppi\iu nj/N}
    =
    \sum_{n=1}^{N}
    a_{n}
    e_{n}(\xi_{j}),
\]
i obtenim
\[
    \widehat{Q}(\xi_{k}) = a_{k},
\]
i per la identitat de Parseval
\[
    \sum_{k=1}^{N}
    \abs{\widehat{Q}(\xi_{k})}^{2}
    =
    \frac{1}{N}
    \sum_{j=1}^{N}
    \abs{P(j/N)}^{2}
    =
    \sum_{k=1}^{N}\abs{a_{k}}^{2}.
\]

\begin{enumerate}
    \item[]\begin{enumerate}
        \item[\textbf{(b)}] Demostreu la fórmula de reconstrucció
            \[
                P(x)
                =
                \sum_{j=1}^{N}
                P(j/N)
                K\bigl(x-(j/N)\bigr)
            \]
            on
            \[
                K(x)
                =
                \frac{\e^{2\uppi\iu x}}{N}
                \frac{1-\e^{2\uppi\iu Nx}}{1-\e^{2\uppi\iu x}}
                =
                \frac{1}{N}
                \bigl(
                    e^{2\uppi\iu x}
                    +
                    \e^{2\uppi\iu2x}
                    + \cdots +
                    \e^{2\uppi\iu Nx}
                \bigr).
            \]
    \end{enumerate}
\end{enumerate}

Tenim que
\begin{align*}
    \sum_{j=1}^{N}
    P(j/N)
    K\bigl(x-(j/N)\bigr)
    &=
    \sum_{j=1}^{N}
    \Biggl(
        \sum_{n=1}^{N}
        a_{n}\e^{2\uppi\iu nj/N}
    \Biggr)
    \Biggl(
        \frac{1}{N}
        \sum_{k=1}^{N}
        \e^{2\uppi\iu k(x-(j/N))}
    \Biggr) \\
    &=
    \sum_{j=1}^{N}
    \Biggl(
        \Biggl(
            \sum_{n=1}^{N}
            a_{n}
            \e^{2\uppi\iu nj/N}
        \Biggr)
        \Biggl(
            \frac{1}{N}
            \sum_{k=1}^{N}
            \e^{2\uppi\iu x}\e^{-2\uppi\iu kj/N}
        \Biggr)
    \Biggr) \\
    &=
    \frac{1}{N}
    \sum_{j=1}^{N}
    \sum_{n=1}^{N}
    \sum_{k=1}^{N}
    a_{n}
    \e^{2\uppi\iu kx}
    \e^{2\uppi\iu(n-k)j/N} \\
    &=
    \frac{1}{N}
    \sum_{n=1}^{N}
    \sum_{k=1}^{N}
    a_{n}\e^{2\uppi kx}
    \sum_{j=1}^{N}
    \e^{2\uppi\iu(n-k)j/N} \\
    &=
    \frac{1}{N}
    \sum_{n=1}^{N}
    \sum_{k=1}^{N}
    a_{n}
    \e^{2\uppi\iu kx}
    \sum_{j=1}^{N}
    e_{n}(\xi_{j})
    \conjugat{e_{k}(\xi_{j})} \\
    &=
    \frac{1}{N}
    \sum_{n=1}^{N}
    \sum_{k=1}^{N}
    a_{n}
    \e^{2\uppi\iu kx}
    \langle e_{n},e_{k}\rangle \\
    &=
    \frac{1}{N}
    \sum_{n=1}^{N}
    a_{n}
    \e^{2\uppi\iu kx}
    N \\
    &=
    \sum_{n=1}^{N}
    a_{n}
    \e^{2\uppi\iu nx}
    = P(x).
\end{align*}

\begin{enumerate}
    \item[\textbf{3.}] Demostreu les següents afirmacions:
        \begin{enumerate}
            \item[\textbf{(a)}] Demostreu que es pot calcular els coeficients de
                Fourier d'una funció a~\(\ZZ(N)\) quan~\(N=3^{n}\) amb, com a
                molt,~\(6N\log_{3}(N)\) operacions.
        \end{enumerate}
\end{enumerate}

\begin{enumerate}
    \item[]\begin{enumerate}
        \item[\textbf{(b)}] Generalitzeu aquest fet per~\(N=\alpha^{n}\)
            on~\(\alpha\) és un enter més gran que~\(1\).
    \end{enumerate}
\end{enumerate}

\begin{enumerate}
    \item[\textbf{4.}] Escriviu les taules de multiplicació dels
        grups~\(\ZZ^{\ast}(3)\),~\(\ZZ^{\ast}(4)\),~\(\ZZ^{\ast}(5)\),~\(\ZZ^{\ast}(6)\),
        \(\ZZ^{\ast}(8)\) i~\(\ZZ^{\ast}(9)\).
        Quins d'aquests grups són cíclics?
\end{enumerate}
\end{document}

